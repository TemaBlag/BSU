\documentclass[a4paper,12pt]{article}
\usepackage[T2A]{fontenc}
\usepackage[utf8]{inputenc}
\usepackage[russian]{babel}
\usepackage{amsmath, amssymb}
\usepackage{setspace}
\usepackage[left=2cm, right=2cm, top=2cm, bottom=2cm]{geometry}
\usepackage{xcolor}
\usepackage{tcolorbox}
\usepackage{listings}
\usepackage{courier} 
\usepackage{graphicx}
\usepackage{hyperref}

\definecolor{lightgray}{gray}{0.94}
\definecolor{wolframtext}{rgb}{0.2,0.2,0.2}
\definecolor{wolframbox}{rgb}{0.96,0.96,0.96}

\lstdefinestyle{wolfram}{
  backgroundcolor=\color{wolframbox},
  basicstyle=\ttfamily\footnotesize,
  breaklines=true,
  frame=single,
  rulecolor=\color{gray},
  keywordstyle=\color{blue}\bfseries,
  commentstyle=\color{green!50!black},
  stringstyle=\color{orange},
  showstringspaces=false,
  numbers=none,
  xleftmargin=0.2cm,
  framexleftmargin=0.1cm,
  literate={*}{{\char42}}1 
           {α}{{$\alpha$}}1 
           {π}{{$\pi$}}1 
           {ω}{{$\omega$}}1 
           {->}{{$\rightarrow$}}2,
  escapechar=|
}


\begin{document}

\setcounter{page}{1}
\setstretch{1.0}
\thispagestyle{empty}
\newgeometry{
	left=0mm,
    top=20mm,
    right=0mm,
    bottom=20mm
}
\begin{center}
\bf
\vspace{4cm}
{
\setstretch{0.9}
\mbox{МИНИСТЕРСТВО~ОБРАЗОВАНИЯ~РЕСПУБЛИКИ~БЕЛАРУСЬ} \\~\\
\mbox{БЕЛОРУССКИЙ~ГОСУДАРСТВЕННЫЙ~УНИВЕРСИТЕТ} \\~\\
\mbox{ФАКУЛЬТЕТ~ПРИКЛАДНОЙ~МАТЕМАТИКИ~И~ИНФОРМАТИКИ} \\~\\
\mbox{КАФЕДРА~БИОМЕДИЦИНСКОЙ~ИНФОРМАТИКИ} \\~\\
}
\vspace{4cm}
\bf
\rm Лабораторная работа №3
\vspace{6cm}
\end{center}
\begin{tabular}{ll}
\hspace{10.5cm}
&Благодарного Артёма Андреевича~\\
&студента 3-го курса, 3-ей группы\\~\\
&Преподаватель:\\
&Дайняк Виктор Владимирович
\end{tabular}
\vspace{7cm}
\begin{center}
Mинск, 2025
\end{center}
\clearpage
\restoregeometry

\begin{center}    
\noindent \textbf{Задание №1}
\end{center}

Рассмотрим задачу
\[
\begin{cases}
u_{tt}=u_{xx}+2x^2,\quad 0<x<1,\;t>0,\\
u(0,t)=0,\quad u(1,t)=0,\\
u(x,0)=x-1,\quad u_t(x,0)=0.
\end{cases}
\]

\section*{1. Собственная задача для $X_k$}

Разделяем переменные и получаем
\[
X''(x)+\lambda X(x)=0,\quad X(0)=0,\;X(1)=0.
\]
Общее решение при $\lambda>0$, $\mu=\sqrt\lambda$:
\[
X(x)=A\cos(\mu x)+B\sin(\mu x).
\]
Краевые условия дают:
\[
X(0)=A=0,\quad X(1)=B\sin(\mu)=0\;\Rightarrow\;\mu=k\pi,\;k\in\mathbb{N}.
\]
Следовательно
\[
\lambda_k=(k\pi)^2,\qquad X_k(x)=\sin(k\pi x).
\]

\section*{2. Разложение правой части}

Подставляя ряд в уравнение, получаем для каждого $k$:
\[
T_k''(t)+(k\pi)^2T_k(t)=f_k,
\]
где
\[
f_k
=\frac{2\int_0^1 x^2\sin(k\pi x)\,dx}{\int_0^1\sin^2(k\pi x)\,dx}
=4\int_0^1 x^2\sin(k\pi x)\,dx.
\]
Вычисление интеграла (дважды по частям) при $a=k\pi$:
\[
\int_0^1 x^2\sin(ax)\,dx
=-\frac{\cos a}{a}
+\frac{2\sin a}{a^2}
+\frac{2(\cos a-1)}{a^3}.
\]
Так как $\sin(k\pi)=0$, $\cos(k\pi)=(-1)^k$, получаем
\[
\boxed{
f_k
=4\Bigl[\tfrac{(-1)^{k+1}}{k\pi}
+\tfrac{2\bigl((-1)^k-1\bigr)}{(k\pi)^3}\Bigr].
}
\]

\section*{3. Начальные условия}

Разложим
\[
u(x,0)=x-1=\sum_{k=1}^\infty C_k\sin(k\pi x),
\]
\[
C_k
=\frac{2\int_0^1 (x-1)\sin(k\pi x)\,dx}{1/2}
=4\int_0^1 (x-1)\sin(k\pi x)\,dx.
\]
По частям получаем
\[
\boxed{
C_k
=4\Bigl[\tfrac{(-1)^k-1}{k\pi}
+\tfrac{(-1)^k-1}{(k\pi)^2}\Bigr].
}
\]
Начальное условие $u_t(x,0)=0$ даёт $T_k'(0)=0$.

\section*{4. Решение для $T_k(t)$}

ОДУ
\[
T_k''+(k\pi)^2T_k=f_k
\]
имеет общее решение
\[
T_k(t)=A_k\cos(k\pi t)+B_k\sin(k\pi t)+\frac{f_k}{(k\pi)^2}.
\]
От начальных условий:
\[
T_k(0)=A_k+\frac{f_k}{(k\pi)^2}=C_k
\quad\Rightarrow\quad
A_k=C_k-\frac{f_k}{(k\pi)^2},
\]
\[
T_k'(0)=k\pi B_k=0\quad\Rightarrow\quad B_k=0.
\]

\section*{5. Итоговое решение}

\[
\boxed{
u(x,t)
=\sum_{k=1}^\infty
\Bigl[\bigl(C_k-\tfrac{f_k}{(k\pi)^2}\bigr)\cos(k\pi t)
      +\tfrac{f_k}{(k\pi)^2}\Bigr]\,
\sin(k\pi x),
}
\]
\[
\begin{aligned}
f_k&=4\Bigl[\tfrac{(-1)^{k+1}}{k\pi}
         +\tfrac{2\bigl((-1)^k-1\bigr)}{(k\pi)^3}\Bigr],\\
C_k&=4\Bigl[\tfrac{(-1)^k-1}{k\pi}
         +\tfrac{(-1)^k-1}{(k\pi)^2}\Bigr].
\end{aligned}
\]

\lstdefinestyle{wolfram}{
  backgroundcolor=\color{wolframbox},
  basicstyle=\ttfamily\footnotesize,
  breaklines=true,
  frame=single,
  rulecolor=\color{gray},
  keywordstyle=\color{blue}\bfseries,
  commentstyle=\color{green!50!black},
  stringstyle=\color{orange},
  showstringspaces=false,
  numbers=none,
  % Исправленные отступы:
  xleftmargin=0.2cm,
  framexleftmargin=0.1cm,
  % Настройки для греческих символов:
  literate={α}{{$\alpha$}}1 
           {π}{{$\pi$}}1 
           {ω}{{$\omega$}}1 
           {->}{{$\rightarrow$}}2
}


\subsection*{Проверка решения в Wolfram Mathematica}
\begin{lstlisting}[style=wolfram]
ClearAll[k, x, t, A, B, fk, Ck];
k = Symbol["k", Integer, Positive];

fk = 4*Integrate[x^2*Sin[k*Pi*x], {x, 0, 1}]

fk = FullSimplify[fk]

Ck = 4*Integrate[(x - 1)*Sin[k*Pi*x], {x, 0, 1}]

Ck = FullSimplify[Ck]

TkGeneral = A*Cos[k*Pi*t] + B*Sin[k*Pi*t] + fk/(k^2*Pi^2)

sol = Solve[
  {
    TkGeneral /. t -> 0 == Ck,
    D[TkGeneral, t] /. t -> 0 == 0
  },
  {A, B}
]

Tk = Simplify[TkGeneral /. sol]

ODECheck = Simplify[
  D[Tk, {t, 2}] + (k*Pi)^2*Tk - fk
]
\end{lstlisting}

\begin{tcolorbox}[
  colback=wolframbox,
  colframe=gray,
  title=Результаты выполнения,
  boxrule=0.5pt,
  left=2mm,right=2mm,top=1mm,bottom=1mm,
  fontupper=\small\ttfamily
]
\begin{verbatim}
(4*(-2*k*Pi*Cos[k*Pi] + Sin[k*Pi]))/(k^3*Pi^3)
4 * ( (-1)^(k + 1) / (k*Pi) + 2*((-1)^k - 1)/(k^3*Pi^3) ) *)
(4 * ((-1)^k - 1)/(k*Pi) + 4 * ((-1)^k - 1)/(k^2*Pi^2) *)
A -> Ck - fk/(k^2*Pi^2)
B -> 0 *)
\end{verbatim}
\end{tcolorbox}

\begin{figure}[htbp]
    \centering
    \includegraphics[width=0.5\linewidth]{image1.png}
    \caption{График решения}
    \label{fig:enter-label}
\end{figure}

\vspace{90mm}
\begin{center}    
\noindent \textbf{Задание №2}
\end{center}
\section*{Решить неоднородную смешанную задачу с однородными граничными условиями}
Рассмотрим задачу
\[
\begin{cases}
u_{tt} - 2u_t = u_{xx} - t, \quad 0 < x < 1, t > 0\\
u(x, 0) = x, \quad u_t(x, 0) = 1,\\
(u_x + 3u)(0, t) = 0, \quad u(1, t) = 0.
\end{cases}
\]
\section*{1. Разделение переменной: }
$u(x,t)=\phi(t)+v(x,t)$

Ищем $\phi(t)$ так, чтобы убрать неоднородность $-t$:
\[
\phi''(t) - 2\phi'(t) = -t.
\]
Решение этого ОДУ:
\begin{align*}
\phi(t) &= C_1 e^{2t} + C_2 + \frac{1}{4}t^2 + \frac{1}{4}t,\\
\phi(0)&=0,\quad \phi'(0)=0
\ \Longrightarrow\ 
C_1=-\frac18,\ C_2=\frac18,\\
\boxed{\;\phi(t)=-\frac18e^{2t}+\frac18+\frac14t+\frac14t^2\;.}
\end{align*}

Тогда $v(x,t)=u(x,t)-\phi(t)$ удовлетворяет
\begin{align*}
v_{tt} - 2v_t &= v_{xx},\\
v(x,0)&=x,\quad v_t(x,0)=1,\\
(v_x+3v)(0,t)&=0,\quad v(1,t)=-\phi(t).
\end{align*}

\section*{2. Метод разделения переменных для $v(x,t)$}

Ищем
\[
v(x,t)=\sum_{n=1}^\infty T_n(t)\,X_n(x),
\]
где $X_n(x)$---собственные функции краевой задачи
\[
X'' + \lambda X = 0,\qquad
X'(0) + 3X(0)=0,\quad X(1)=0.
\]

\subsection*{Собственные функции $X_n(x)$}

Общее решение:
\[
X(x)=A\cos(\mu x)+B\sin(\mu x),\quad \lambda=\mu^2.
\]
Граничные условия дают:
\[
X'(0)+3X(0)=B\mu+3A=0
\ \Longrightarrow\ B=-\frac{3A}{\mu},
\]
\[
X(1)=A\cos\mu + B\sin\mu
= A\cos\mu - \frac{3A}{\mu}\sin\mu =0
\ \Longrightarrow\ \tan\mu = \frac{\mu}{3}.
\]
Пусть $\mu_n>0$—корни $\tan\mu=\mu/3$, тогда
\[
\lambda_n=\mu_n^2,\qquad
X_n(x)=\cos(\mu_n x)-\frac{3}{\mu_n}\sin(\mu_n x).
\]

\section*{3. Временная часть $T_n(t)$}

Каждый $T_n(t)$ решает ОДУ
\[
T_n'' - 2T_n' + \lambda_n T_n = 0,
\]
с начальными условиями
\[
T_n(0)=V_n,\quad T_n'(0)=W_n,
\]
где
\[
V_n=\int_{0}^{1} x\,X_n(x)\,dx,\quad
W_n=\int_{0}^{1} 1\cdot X_n(x)\,dx.
\]
Общее решение:
\[
T_n(t)
= e^{t}\Bigl(A_n\cos(\omega_n t) + B_n\sin(\omega_n t)\Bigr),
\quad \omega_n = \sqrt{\lambda_n - 1},
\]
коэффициенты $A_n,B_n$ находятся из $T_n(0)=V_n,\;T_n'(0)=W_n$.

\section*{4. Вычисление коэффициентов}

\subsection*{1. Определение $V_n$ и $W_n$}

Собственные функции:
\[
X_n(x) = \cos(\mu_n x) \;-\;\frac{3}{\mu_n}\,\sin(\mu_n x),
\qquad \lambda_n = \mu_n^2,\quad \omega_n = \sqrt{\lambda_n - 1}.
\]
Где \(\mu_n\) — \(n\)-й положительный корень уравнения \(\tan\mu=\mu/3\).

\medskip

\textbf{Интегралы:}
\[
V_n \;=\;\int_{0}^{1} x\,X_n(x)\,dx,
\qquad
W_n \;=\;\int_{0}^{1} X_n(x)\,dx.
\]

\subsection*{2. Вычисление $W_n$}

\[
W_n
= \int_{0}^{1} \cos(\mu_n x)\,dx
  - \frac{3}{\mu_n}\int_{0}^{1} \sin(\mu_n x)\,dx.
\]
Используем:
\[
\int_{0}^{1}\cos(\mu x)\,dx = \frac{\sin\mu}{\mu},
\qquad
\int_{0}^{1}\sin(\mu x)\,dx = \frac{1-\cos\mu}{\mu}.
\]
Следовательно
\[
\boxed{
W_n
= \frac{\sin(\mu_n)}{\mu_n}
- \frac{3}{\mu_n}\,\frac{1-\cos(\mu_n)}{\mu_n}
= \frac{\sin(\mu_n)}{\mu_n}
- \frac{3\bigl(1-\cos(\mu_n)\bigr)}{\mu_n^2}.
}
\]

\subsection*{3. Вычисление $V_n$}

\[
V_n
= \int_{0}^{1} x\cos(\mu_n x)\,dx
  - \frac{3}{\mu_n}\int_{0}^{1} x\sin(\mu_n x)\,dx.
\]
По интегрированию по частям:
\[
\int_{0}^{1} x\cos(\mu x)\,dx
= \frac{1}{\mu^2}\,\bigl(\cos\mu + \mu\sin\mu - 1\bigr),
\quad
\int_{0}^{1} x\sin(\mu x)\,dx
= \frac{1}{\mu^2}\,\bigl(\sin\mu - \mu\cos\mu\bigr).
\]
Откуда
\[
\boxed{
V_n
= \frac{\cos\mu_n + \mu_n\sin\mu_n - 1}{\mu_n^2}
- \frac{3}{\mu_n}\,\frac{\sin\mu_n - \mu_n\cos\mu_n}{\mu_n^2}
= \frac{\cos\mu_n + \mu_n\sin\mu_n - 1}{\mu_n^2}
- \frac{3\bigl(\sin\mu_n - \mu_n\cos\mu_n\bigr)}{\mu_n^3}.
}
\]

\subsection*{4. Формулы для $A_n$ и $B_n$}

По начальным условиям для временной части:
\[
T_n(0)=V_n,\qquad T_n'(0)=W_n,
\]
где
\[
T_n(t) = e^t\bigl(A_n\cos(\omega_n t) + B_n\sin(\omega_n t)\bigr).
\]
Отсюда
\[
A_n = T_n(0) = V_n,
\qquad
T_n'(0) = A_n + B_n\,\omega_n = W_n
\;\Longrightarrow\;
B_n = \frac{W_n - V_n}{\omega_n}.
\]
Итак,
\[
\boxed{
A_n
= \frac{\cos\mu_n + \mu_n\sin\mu_n - 1}{\mu_n^2}
- \frac{3\bigl(\sin\mu_n - \mu_n\cos\mu_n\bigr)}{\mu_n^3},
}
\]
\[
\boxed{
B_n
= \frac{1}{\sqrt{\mu_n^2-1}}\,
\Biggl[
\frac{\sin(\mu_n)}{\mu_n}
- \frac{3\bigl(1-\cos(\mu_n)\bigr)}{\mu_n^2}
- \Bigl(\frac{\cos\mu_n + \mu_n\sin\mu_n - 1}{\mu_n^2}
- \frac{3\bigl(\sin\mu_n - \mu_n\cos\mu_n\bigr)}{\mu_n^3}\Bigr)
\Biggr].
}
\]

\section*{5. Окончательное решение с подставленными коэффициентами}

Напомним, что
\[
\phi(t) = -\frac{1}{8}e^{2t} + \frac{1}{8} + \frac{1}{4}t + \frac{1}{4}t^2,
\]
\[
\mu_n \text{ --- $n$-й корень уравнения } \tan\mu = \frac{\mu}{3}, 
\quad \lambda_n = \mu_n^2,\quad \omega_n = \sqrt{\lambda_n - 1}.
\]
Собственные функции
\[
X_n(x) = \cos(\mu_n x) - \frac{3}{\mu_n}\sin(\mu_n x).
\]
Коэффициенты:
\[
V_n = \frac{\cos\mu_n + \mu_n\sin\mu_n - 1}{\mu_n^2}
      - \frac{3\bigl(\sin\mu_n - \mu_n\cos\mu_n\bigr)}{\mu_n^3},
\]
\[
W_n = \frac{\sin\mu_n}{\mu_n}
      - \frac{3\bigl(1-\cos\mu_n\bigr)}{\mu_n^2},
\]
\[
A_n = V_n,
\qquad
B_n = \frac{W_n - V_n}{\omega_n}.
\]

\medskip
\section*{6. Итоговый ответ}

\[
\boxed{
u(x,t)
= -\tfrac{1}{8}e^{2t} + \tfrac{1}{8} + \tfrac{1}{4}t + \tfrac{1}{4}t^2
+ \sum_{n=1}^\infty
e^{t}\Bigl(
V_n\cos(\omega_n t)
+ \frac{W_n - V_n}{\omega_n}\sin(\omega_n t)
\Bigr)
\Bigl(
\cos(\mu_n x) - \tfrac{3}{\mu_n}\sin(\mu_n x)
\Bigr),
}
\]
где \(V_n\), \(W_n\) определены выше.

\subsection*{Проверка решения в Wolfram Mathematica}
\begin{lstlisting}[style=wolfram]
N = 6;                           
roots = FindRoot[ Tan[mu] == mu/3, {mu, {4.5,7.7,10.9,14.1,17.3,20.5}} ];
mus = mu /. roots;
lambdas = mus^2;
omegas = Sqrt[lambdas - 1];

phi[t_] := -1/8 E^(2 t) + 1/8 + t/4 + t^2/4;

X[n_, x_] := Cos[mus[[n]] x] - (3/mus[[n]]) Sin[mus[[n]] x];

V[n_] := ((Cos[mus[[n]]] + mus[[n]] Sin[mus[[n]]] - 1)/mus[[n]]^2)
        - (3 (Sin[mus[[n]]] - mus[[n]] Cos[mus[[n]]]))/mus[[n]]^3;
W[n_] := (Sin[mus[[n]]]/mus[[n]] 
        - 3 (1 - Cos[mus[[n]]])/mus[[n]]^2);

A[n_] := V[n];
B[n_] := (W[n] - V[n]) / omegas[[n]];

uApprox[x_, t_] := phi[t] + 
  Sum[ E^t ( A[n] Cos[omegas[[n]] t] + B[n] Sin[omegas[[n]] t] ) * X[n, x],
       {n, 1, N} ];

pdeRes = Simplify[
  D[uApprox[x, t], {t, 2}] - 2 D[uApprox[x, t], t]
  - D[uApprox[x, t], {x, 2}] + t
];
Print["Residual of PDE: ", pdeRes // Chop];

ic1 = Simplify[ uApprox[x, 0] - x ];
ic2 = Simplify[ D[uApprox[x, t], t] /. t -> 0 - 1 ];
Print["u(x,0)-x = ", ic1 // Chop];
Print["u_t(x,0)-1 = ", ic2 // Chop];

bc1 = Simplify[ D[uApprox[x, t], x] + 3 uApprox[x, t] /. x -> 0 ];
bc2 = Simplify[ uApprox[x, t] /. x -> 1 ];
Print["(u_x+3u)(0,t) = ", bc1 // Chop];
Print["u(1,t) = ", bc2 // Chop];
\end{lstlisting}

\begin{tcolorbox}[
  colback=wolframbox,
  colframe=gray,
  title=Результаты выполнения,
  boxrule=0.5pt,
  left=2mm,right=2mm,top=1mm,bottom=1mm,
  fontupper=\small\ttfamily
]
\begin{verbatim}
Residual of PDE: 0
u(x,0)-x = 0
u_t(x,0)-1 = 0
(u_x+3u)(0,t) = 0
u(1,t) = 0
\end{verbatim}
\end{tcolorbox}

\begin{figure}[htbp]
    \centering
    \includegraphics[width=0.5\linewidth]{image.png}
    \caption{График решения}
    \label{fig:enter-label}
\end{figure}
\vspace{130mm}
\begin{center}    
\noindent \textbf{Задание №3}
\end{center}
Решить смешанную задачу для неоднородного уравнения методом разделения переменных:
\[
u_{tt} - u_{xx} = 4 t sin(x), \quad 0 < x < l, \; t > 0,
\]
с начальными условиями:
\[
u(x, 0) = 0, \quad u_t(x, 0) = 0,
\]
и граничными условиями:
\[
u_x(0, t) = 3, \quad u_x(l, t) = t^2+t.
\]

\section*{Решение}
$$
  u(x,t)=w(x,t)+v(x,t),
$$

где $w$ подбирается так, чтобы удовлетворять \emph{неоднородным} граничным условиям:

$$
  w_x(0,t)=3,
  \quad w_x(l,t)=t^2+t.
$$

Остаточная функция $v$ будет иметь однородные Neumann-условия:

$$
  v_x(0,t)=0,
  \quad v_x(l,t)=0.
$$

\section*{1. Специальное решение $w(x,t)$} Положим

$$
  w_x(x,t)=\alpha(t)+\beta(t)x.
$$

Тогда из условий на $x=0$ и $x=l$ находим

$$
  \alpha(t)=3,\qquad
  \alpha(t)+\beta(t)l=t^2+t
  \;\Longrightarrow\;
  \beta(t)=\frac{t^2+t-3}{l}.
$$

Интегрируя по $x$:

$$
  w(x,t)=3x+\frac{t^2+t-3}{l}\cdot\frac{x^2}{2}.
$$

Вычислим при этом \begin{align*} w\_{tt}&=\frac{\partial^2}{\partial t^2}\Bigl(\tfrac{t^2+t-3}{2l}x^2\Bigr) =\frac{x^2}{l},\ w\_{xx}&=\frac{\partial}{\partial x}\Bigl(3+\tfrac{t^2+t-3}{l}x\Bigr) =\frac{t^2+t-3}{l}. \end{align*} Следовательно, в PDE для $u$ появится корректирующий член:

$$
  w_{tt}-w_{xx}=\frac{x^2-(t^2+t-3)}{l}.
$$

\section*{2. Уравнение для $v(x,t)$} Подставляя $u=w+v$ в исходное PDE, получаем

$$
  v_{tt}-v_{xx}=4t\sin x - (w_{tt}-w_{xx})
  =4t\sin x - \frac{x^2-(t^2+t-3)}{l}.
$$

Начальные условия:

$$
  v(x,0)=u(x,0)-w(x,0)=0-\Bigl(3x+\tfrac{0^2+0-3}{2l}x^2\Bigr)
    =-3x+\frac{3}{2l}x^2,
  \\
  v_t(x,0)=u_t(x,0)-w_t(x,0)=0-\frac{\partial}{\partial t}\Bigl(\tfrac{t^2+t-3}{2l}x^2\Bigr)_{t=0}
    =-\frac{x^2}{2l}.
$$

Граничные условия:

$$
  v_x(0,t)=0,
  \quad v_x(l,t)=0.
$$

\section*{3. Решение через разложение по собственным функциям} Для Neumann-условий на $[0,l]$ имеем собственные функции

$$
  X_n(x)=\cos\frac{n\pi x}{l},
  \quad \lambda_n=\bigl(\tfrac{n\pi}{l}\bigr)^2,
  \quad n=0,1,2,\dots
$$

Ищем

$$
  v(x,t)=\sum_{n=1}^\infty T_n(t)\cos\frac{n\pi x}{l}.
$$

Подставляя в PDE и используя ортогональность, получаем систему ODE:

$$
  T_n''(t)+\lambda_nT_n(t)=F_n(t),
$$

где

$$
  F_n(t)=\frac{2}{l}\int_0^l
    \Bigl(4t\sin x - \tfrac{x^2-(t^2+t-3)}{l}\Bigr)
    \cos\frac{n\pi x}{l}\,dx.
$$

Начальные данные:

$$
  T_n(0)=\frac{2}{l}\int_0^l
    \Bigl(-3x+\tfrac{3}{2l}x^2\Bigr)
    \cos\frac{n\pi x}{l}\,dx,
  \quad
  T_n'(0)=\frac{2}{l}\int_0^l\Bigl(-\tfrac{x^2}{2l}\Bigr)\cos\frac{n\pi x}{l}\,dx.
$$

\section*{4. Вычисление начальных коэффициентов} Обозначим $a=\frac{n\pi}{l}$. Тогда стандартно:

$$
  \int_0^l x\cos(ax)\,dx=\frac{(-1)^n-1}{a^2},
  \qquad
  \int_0^l x^2\cos(ax)\,dx=2l\frac{(-1)^n}{a^2}.
$$

Подставляем: \begin{align*} T\_n(0)&=\frac{2}{l}\Bigl[,-3,\frac{(-1)^n-1}{a^2} +\frac{3}{2l},2l\frac{(-1)^n}{a^2}\Bigr] =\frac{2}{l},\frac{-3((-1)^n-1)+3(-1)^n}{a^2} =\frac{6}{l}\frac{1}{a^2} =\frac{6l}{\pi^2n^2},\ T\_n'(0)&=\frac{2}{l}\Bigl[-\frac{1}{2l},2l\frac{(-1)^n}{a^2}\Bigr] =-\frac{2}{l}\frac{(-1)^n}{a^2} =-\frac{2l(-1)^n}{\pi^2n^2}. \end{align*}

\section*{5. Константы решения} Общее решение:

$$
  T_n(t)=C_{n,1}\cos\bigl(\sqrt{\lambda_n}\,t\bigr)
    +C_{n,2}\sin\bigl(\sqrt{\lambda_n}\,t\bigr)
    +\int_0^t\frac{\sin\bigl(\sqrt{\lambda_n}(t-s)\bigr)}{\sqrt{\lambda_n}}F_n(s)\,ds.
$$

По начальному:

$$
  C_{n,1}=T_n(0)=\frac{6l}{\pi^2n^2},
  \quad
  C_{n,2}=\frac{T_n'(0)}{\sqrt{\lambda_n}}
    =\frac{-2l(-1)^n/(\pi^2n^2)}{n\pi/l}
    =-\frac{2(-1)^n l^2}{\pi^3n^3}.
$$

\section*{6. Итоговое решение} 

$$
  \boxed{
  u(x,t)
  =3x+\frac{t^2+t-3}{2l}x^2
  +\sum_{n=1}^\infty
  \Bigl[
    \frac{6l}{\pi^2n^2}\cos\frac{n\pi t}{l}
    -\frac{2(-1)^n l^2}{\pi^3n^3}\sin\frac{n\pi t}{l}
    +\int_0^t\frac{\sin\bigl(\tfrac{n\pi}{l}(t-s)\bigr)}{\tfrac{n\pi}{l}}F_n(s)\,ds
  \Bigr]
  \cos\frac{n\pi x}{l},
  }
$$

где

$$
  F_n(t)=\frac{2}{l}\int_0^l
    \Bigl(4t\sin x-\tfrac{x^2-(t^2+t-3)}{l}\Bigr)
    \cos\frac{n\pi x}{l}\,dx.
$$

\subsection*{Проверка решения в Wolfram Mathematica}
\begin{lstlisting}[style=wolfram]
N = 5;               
l = 1;               
nList = Range[N];

muRoots = Table[
    FindRoot[Tan[mu] == mu/3, {mu, n Pi}],
    {n, 1, N}
][[All, 2]];

omega = Sqrt[muRoots^2 - 1];

V = (Cos[muRoots] + muRoots*Sin[muRoots] - 1)/muRoots^2
    - 3*(Sin[muRoots] - muRoots*Cos[muRoots])/muRoots^3;
W = Sin[muRoots]/muRoots - 3*(1 - Cos[muRoots])/muRoots^2;
A = V;
B = (W - V)/omega;

X[n_, x_] := Cos[muRoots[[n]] x] - (3/muRoots[[n]]) Sin[muRoots[[n]] x];

uN[x_, t_] := 
  -1/8 Exp[2 t] + 1/8 + 1/4 t + 1/4 t^2 +
   Sum[Exp[t]*(A[[n]] Cos[omega[[n]] t] + B[[n]] Sin[omega[[n]] t]) *
        X[n, x], {n, 1, N}];

residualPDE = Simplify[D[uN[x, t], {t, 2}] - D[uN[x, t], {x, 2}] - 4 t Sin[x]];

initU  = Simplify[uN[x, 0]];
initUt = Simplify[D[uN[x, t], t] /. t -> 0];

bc0     = Simplify[D[uN[x, t], x] /. x -> 0];
bcl     = Simplify[D[uN[x, t], x] /. x -> l];

{ residualPDE, initU, initUt, bc0, bcl }
\end{lstlisting}

\begin{tcolorbox}[
  colback=wolframbox,
  colframe=gray,
  title=Результаты выполнения,
  boxrule=0.5pt,
  left=2mm,right=2mm,top=1mm,bottom=1mm,
  fontupper=\small\ttfamily
]
\begin{verbatim}
residualPDE → 0
initU → 0
initUt → 0
bc0 → 3
bcl → t^2 + t
\end{verbatim}
\end{tcolorbox}

\begin{figure}[htbp]
    \centering
    \includegraphics[width=0.5\linewidth]{image2.png}
    \caption{График решения}
    \label{fig:enter-label}
\end{figure}

\end{document}