\documentclass[a4paper,12pt]{article}
\usepackage[T2A]{fontenc}
\usepackage[utf8]{inputenc}
\usepackage[russian]{babel}
\usepackage{amsmath, amssymb}
\usepackage{setspace}
\usepackage[left=2cm, right=2cm, top=2cm, bottom=2cm]{geometry}

\begin{document}

\setcounter{page}{1}
\setstretch{1.0}
\thispagestyle{empty}
\newgeometry{
	left=0mm,
    top=20mm,
    right=0mm,
    bottom=20mm
}
\begin{center}
\bf
\vspace{4cm}
{
\setstretch{0.9}
\mbox{МИНИСТЕРСТВО~ОБРАЗОВАНИЯ~РЕСПУБЛИКИ~БЕЛАРУСЬ} \\~\\
\mbox{БЕЛОРУССКИЙ~ГОСУДАРСТВЕННЫЙ~УНИВЕРСИТЕТ} \\~\\
\mbox{МЕХАНИКО-МАТЕМАТИЧЕСКИЙ~ФАКУЛЬТЕТ} \\~\\
\mbox{Кафедра~биомедицинской~информатики} \\~\\
}
\vspace{4cm}
\bf
\mbox{Классификация дифференциальных уравнений с частными производными}\\
\vspace{1cm}
\rm Лабораторная работа 
\vspace{3cm}
\end{center}
\begin{tabular}{ll}
\hspace{10.5cm}
&Благодарного Артёма Андреевича~\\
&студента 3-го курса\\~\\
&Преподаватель:\\
&Дайняк Виктор Владимирович
\end{tabular}
\vspace{7cm}
\begin{center}
Mинск, 2025
\end{center}
\clearpage
\restoregeometry

\begin{center}    
\noindent \textbf{Задание 1. №1.5}
\end{center}

Привести к одному из канонических видов уравнение:

\begin{equation*}
    3u_{xx} + u_{xy} + 3u_{x} + u_{y} - u = -y
\end{equation*}

\textbf{Решение:}

Характеристическое уравнение:
\begin{equation*}
    3 (dy)^2 - dy \cdot dx = 0
\end{equation*}
\begin{equation*}
   3 \left( \frac{dy}{dx} \right)^2 - \left( \frac{dy}{dx} \right) = 0
\end{equation*}

Подстановка $t = \frac{dy}{dx}$:
\begin{equation*}
    3t^2 - t = 0 
\end{equation*}

Дискриминант $D = 1 > 0$, следовательно, уравнение \textbf{гиперболического типа}.

Решение уравнения:
\begin{equation*}
    t(3t - 1) = 0 \Rightarrow 
    \begin{cases}
    t_1 = 0 \\
    t_2 = \frac{1}{3}
    \end{cases}
\end{equation*}

Соответствующие характеристические переменные:
\begin{align*}
    &t_1 = 0: \quad y = C_1 \\
    &t_2 = \frac{1}{3}: \quad y = \frac{1}{3} x + C_2
\end{align*}

Выбираем новые переменные:
\begin{align*}
    \xi &= y, \quad \quad \quad \xi_x = 0,  \quad \quad \xi_y = 1 \\
    \eta &= y - \frac{x}{3}, \quad \eta_x = -\frac{1}{3},  \quad \eta_y = 1
\end{align*}

Преобразование производных:
\begin{flushleft}
\(
\begin{array}{r|l}
3 & u_x = -\frac{1}{3} u_{\eta} \\
1 & u_y = u_{\eta} + u_{\xi} \\
3 & u_{xx} = \frac{1}{9} u_{\eta\eta} \\
1 & u_{xy} = -\frac{1}{3} u_{\eta\xi} - \frac{1}{3} u_{\eta\eta}
\end{array}
\)
\end{flushleft}

Пересчёт коэффициентов:
\begin{flushleft}
\(
\begin{array}{rcl}
u_{\xi} & : & 1 \\
u_{\eta} & : & -1+1=0 \\
u_{\xi\xi} & : & 0\\
u_{\xi\eta} & : & -\frac{1}{3}\\
u_{\eta\eta} & : &  \frac{1}{3} -  \frac{1}{3} = 0
\end{array}
\)
\end{flushleft}

\begin{flushleft}
\begin{flalign*}
&-\frac{1}{3} u_{\xi\eta} + u_{\xi} - u = -\xi &\\
&u_{\xi\eta} - 3 u_{\xi} + 3 u - 3 \xi = 0 &
\end{flalign*}
\text{Ответ: } \( u_{\xi\eta} - 3 u_{\xi} + 3 u - 3 \xi = 0. \)
\end{flushleft}

\begin{center}    
\noindent \textbf{Задание 2. №1.29}
\end{center}

Привести к каноническому виду уравнение:

\begin{equation*}
    u_{xx} - 2 \sin x \cdot u_{xy} - \cos^2 x \cdot u_{yy} - \cos x \cdot u_y = 0
\end{equation*}

\textbf{Решение:}

Характеристическое уравнение:
\begin{equation*}
    {(dy)}^2 + 2 \sin x \, {dy}{dx} - \cos^2 x \, {(dx)}^2 = 0
\end{equation*}
\begin{equation*}
    \left( \frac{dy}{dx} \right)^2 + 2 \sin x \frac{dy}{dx} - \cos^2 x = 0
\end{equation*}

Подстановка $t = \frac{dy}{dx}$:
\begin{equation*}
    t^2 + 2 \sin x \, t - \cos^2 x \,= 0 
\end{equation*}

Дискриминант $D = 4 > 0$, следовательно, уравнение \textbf{гиперболического типа}.

Решение квадратного уравнения:

\begin{equation*}
    t_{1,2} = - \sin x \pm 1
\end{equation*}

Соответствующие характеристики:
\begin{align*}
    &t_1 = - \sin x - 1: \quad y = \cos x - x + C_1 \\
    &t_2 = - \sin x + 1: \quad y = \cos x + x + C_2
\end{align*}

Выбираем новые переменные:
\begin{align*}
     \xi &= y - \cos x + x,  \quad \quad \quad \xi_x = \sin x + 1 \, ,  \quad \xi_y = 1, \quad \xi_{xx} = \cos x\\
     \eta &= y - \cos x - x, \quad \quad \quad \eta_x = \sin x - 1 \, ,  \quad \eta_y = 1, \quad \eta_{xx} = \cos x
\end{align*}

Преобразование производных:
\begin{flushleft}
\(
\begin{array}{r|l}
0 & u_x = (\sin x + 1)u_{\xi} + (\sin x - 1)u_{\eta} \\
- \cos x \, & u_y = u_{\eta} + u_{\xi} \\
1 & u_{xx} = (\sin x + 1)^2u_{\xi\xi} + 2(\sin x + 1)(\sin x - 1)u_{\xi\eta} + (\sin x - 1)^2u_{\eta\eta} + \\ & + \cos x \,u_{\xi} + \cos x \,u_{\eta}\\
- 2 \sin x \, & u_{xy} = (\sin x + 1) u_{\xi\xi} + 2 \sin x \, u_{\xi\eta} + (\sin x - 1) u_{\eta\eta} \\
- \cos^2 x \, & u_{yy} = u_{\xi\xi} + 2u_{\xi\eta} + u_{\eta\eta} \\
\end{array}
\)
\end{flushleft}

Пересчёт коэффициентов:
\begin{flushleft}
\(
\begin{array}{rcl}
u_{\xi} & : & -\cos x \, + \cos x \, = 0\\
u_{\eta} & : & -\cos x \, + \cos x \, = 0 \\
u_{\xi\xi} & : & (\sin x + 1)^2 - 2 \sin x \, (\sin x \, + 1) - \cos^2 x \, = 0 \\
u_{\xi\eta} & : & 2(\sin x + 1)(\sin x - 1) - 4 \sin^2 x \, - 2 \cos^2 x \,  = -4\\
u_{\eta\eta} & : &  (\sin x - 1)^2 - 2 \sin x \, (\sin x \, - 1) - \cos^2 x \, = 0
\end{array}
\)
\end{flushleft}

$-4 u_{\xi\eta} = 0$

\vspace{2mm}
\textbf{Ответ:} $ u_{\xi\eta} = 0. $

\vspace{14mm}
\begin{center}    
\noindent \textbf{Задание 3. №1.50}
\end{center}

Привести к каноническому виду уравнение в каждой из областей, где сохраняется тип уравнения:
\begin{equation*}
    \operatorname{sign} y \, u_{xx} + 2 u_{xy} + u_{yy} = 0
\end{equation*}

\textbf{Решение:}

Характеристическое уравнение:
\begin{equation*}
    \operatorname{sign} y (dy)^2 - 2 dy dx + (dx)^2 = 0
\end{equation*}
\begin{equation*}
    \operatorname{sign} y \left( \frac{dy}{dx} \right)^2 - 2 \frac{dy}{dx} + 1 = 0
\end{equation*}


Подстановка $t = \frac{dy}{dx}$:
\begin{equation*}
    \operatorname{sign} y \cdot t^2 - 2t + 1 = 0
\end{equation*}

Дискриминант:
\begin{equation*}
    D = 4 - 4 \operatorname{sign} y
\end{equation*}
\begin{flushleft}
    \(\text{I.} \quad \operatorname{sign} y = 0, \quad y = 0, \quad D = 4 > 0, \text{ следовательно, уравнение \textbf{гиперболического типа}.}\)
\end{flushleft}
\begin{equation*}
    \begin{aligned}
        - 2 dy \, dx + (dx)^2 &= 0 \\
        dx(dx - 2 dy) &= 0 \\
    \end{aligned}
\end{equation*}

Соответствующие характеристики:
\begin{align*}
    &dx = 0 \quad \quad x = C_1 \\
    &dy = \frac{1}{2} dx \quad y = \frac{1}{2}  x + C_2
\end{align*}
Выбираем новые переменные:
\begin{align*}
    \xi &= y - \frac{1}{2}x, \quad \xi_x = -\frac{1}{2}, \quad \xi_y = 1 \\
    \eta &= x, \quad  \quad  \quad  \quad \eta_x = 1,  \quad \quad \eta_y = 0
\end{align*}


Преобразование производных:
\begin{flushleft}
\(
\begin{array}{r|l}
0 & u_x = - \frac{1}{2}u_{\xi} + u_{\eta} \\
0 & u_y = u_{\xi}\\
0 & u_{xx} = \frac{1}{4}u_{\xi\xi} - u_{\xi\eta} + u_{\eta\eta}\\
2 & u_{xy} = -\frac{1}{2}u_{\xi\xi} + u_{\xi\eta}\\
1 & u_{yy} = u_{\xi\xi}
\end{array}
\)
\end{flushleft}

Пересчёт коэффициентов:
\begin{flushleft}
\(
\begin{array}{rcl}
u_{\xi} & : &  0\\
u_{\eta} & : & 0 \\
u_{\xi\xi} & : &  1 - 1 = 0\\
u_{\xi\eta} & : & 2 \\
u_{\eta\eta} & : &  0
\end{array}
\)
\end{flushleft}

$2 u_{\xi\eta} = 0$

\textbf{Ответ:} $ u_{\xi\eta} = 0 $

\begin{flushleft}
    \(\text{II.} \quad \operatorname{sign} y = 1, \quad y > 0, \quad D = 0, \text{ следовательно, уравнение \textbf{параболического типа}.}\)
\end{flushleft}

Характеристическое уравнение:
\begin{align*}
    t^2 - 2t + 1 = 0 \\
    (t - 1)^2 = 0 \\
    t = 1
\end{align*}

Характеристики:
\begin{align*}
    &dy = dx \quad y = x + C_1 \\
    &x = C_2
\end{align*}

Выбираем новые переменные:
\begin{align*}
    \xi &= y - x \quad \quad \xi_x = -1, \quad \xi_y = 1 \\
    \eta &= x \quad  \quad  \quad  \quad \eta_x = 1,  \quad \quad \eta_y = 0
\end{align*}

Преобразование производных:
\begin{flushleft}
\(
\begin{array}{r|l}
0 & u_x = - u_{\xi} + u_{\eta} \\
0 & u_y = u_{\xi}\\
1 & u_{xx} = u_{\xi\xi} - 2 u_{\xi\eta} + u_{\eta\eta}\\
2 & u_{xy} = -u_{\xi\xi} + u_{\xi\eta}\\
1 & u_{yy} = u_{\xi\xi}
\end{array}
\)
\end{flushleft}

Пересчёт коэффициентов:
\begin{flushleft}
\(
\begin{array}{rcl}
u_{\xi} & : &  0\\
u_{\eta} & : & 0 \\
u_{\xi\xi} & : &  1 + 1 - 2 = 0\\
u_{\xi\eta} & : & -2 + 2 = 0 \\
u_{\eta\eta} & : &  1
\end{array}
\)
\end{flushleft}

$u_{\eta\eta} = 0$

\textbf{Ответ:} $ u_{\eta\eta} = 0. $

\begin{flushleft}
    \(\text{III.} \quad \operatorname{sign} y = -1, \quad y < 0, \quad D = 8, \text{ следовательно, уравнение \textbf{гиперболического типа}.}\)
\end{flushleft}

Характеристическое уравнение:
\begin{equation*}
    -t^2 - 2t + 1 = 0
\end{equation*}

Решение квадратного уравнения:
\begin{equation*}
    t_{1,2} = \frac{2 \pm \sqrt{8}}{-2} = -1 \pm \sqrt{2}
\end{equation*}

Соответствующие характеристически:
\begin{align*}
    dy = (-1 - \sqrt{2})dx \quad  \quad y = (-1 - \sqrt{2})x + C_1 \\
    dy = (-1 + \sqrt{2})dx \quad  \quad y = (-1 + \sqrt{2})x + C_2
\end{align*}

Выбираем новые переменные:
\begin{align*}
    \xi &= y + (1 + \sqrt{2}) x, \quad \quad \xi_x = 1 + \sqrt{2}, \quad \quad \xi_y = 1 \\
    \eta &= y + (1 - \sqrt{2}) x, \quad  \quad\eta_x = 1 - \sqrt{2}, \quad \quad \eta_y = 1
\end{align*}

Преобразование производных:
\begin{flushleft}
\(
\begin{array}{r|l}
0 & u_x = (1 + \sqrt{2}) u_{\xi} + (1 - \sqrt{2}) u_{\eta} \\
0 & u_y = u_{\xi} + u_{\eta}\\
-1 & u_{xx} = (1 + \sqrt{2})^2 u_{\xi\xi} + 2(1 + \sqrt{2})(1 - \sqrt{2})u_{\xi\eta} + (1 - \sqrt{2})^2u_{\eta\eta}\\
2 & u_{xy} = (1 + \sqrt{2})u_{\xi\xi} + 2u_{\xi\eta} + (1 - \sqrt{2})u_{\eta\eta}\\
1 & u_{yy} = u_{\xi\xi} + 2u_{\xi\eta} + u_{\eta\eta}
\end{array}
\)
\end{flushleft}

Пересчёт коэффициентов:
\noindent
\begin{flalign*}
    u_{\xi} &: 0 &\\
    u_{\eta} &: 0 &\\
    u_{\xi\xi} &: - (1+\sqrt{2})^2 + 2(1+\sqrt{2})(1-\sqrt{2}) + 1 = 0 &\\
    u_{\eta\eta} &: - (1-\sqrt{2})^2 + 2(1+\sqrt{2})(1-\sqrt{2}) + 1 = 0 &\\
    u_{\xi\eta} &: -2(1+\sqrt{2})(1-\sqrt{2}) + 4 + 2 = 8 &
\end{flalign*}

$8u_{\xi\eta} = 0$

\textbf{Ответ:} $ u_{\xi\eta} = 0. $

\begin{center}
\noindent \textbf{Канонический вид уравнения}
\end{center}

\[
\operatorname{sign} y \cdot u_{xx} + 2 u_{xy} + u_{yy} = 0
\]

\begin{equation*}
    \begin{cases}
        u_{\xi\eta} = 0, \quad y = 0, \quad D = 4 \\
        u_{\xi\eta} = 0, \quad y < 0, \quad D = 8 \\
        u_{\eta\eta} = 0, \quad y > 0, \quad D = 0
    \end{cases}
\end{equation*}


\vspace{100mm}
\vspace{17mm}
\begin{center}    
\noindent \textbf{Задание 4. №5}
\end{center}
Привести к каноническому виду и упростить уравнение:

\begin{equation*}
    a u_{xx} + 2a u_{xy} + a u_{yy} + b u_x + c u_y + u = 0
\end{equation*}

\textbf{Решение:}

Характеристическое уравнение:
\begin{equation*}
    a dy^2 - 2a dy dx + a (dx)^2 = 0
\end{equation*}
\begin{equation*}
    a \left(\frac{dy}{dx}\right)^2 - 2a \left(\frac{dy}{dx}\right) + a = 0
\end{equation*}

Подстановка $t = \frac{dy}{dx}$:
\begin{equation*}
    a t^2 - 2a t + a = 0
\end{equation*}

Дискриминант $D = 4a^2 - 4a^2 = 0$, следовательно, уравнение \textbf{параболического типа}.

Решение квадратного уравнения:
\begin{equation*}
    a (t - 1)^2 = 0
\end{equation*}
\begin{equation*}
    t = 1
\end{equation*}

Соответствующие характеристики:
\begin{equation*}
    dy = dx
\end{equation*}
\begin{equation*}
    y = x + C_1
\end{equation*}

Выбираем новые переменные:
\begin{align*}
    \xi &= y - x,   \quad \xi_x = -1, \quad \quad \xi_y = 1 \\
    \eta &= x, \quad \quad \quad \eta_x = 1, \quad \quad \eta_y = 0
\end{align*}

Преобразование производных:
\begin{flushleft}
\(
\begin{array}{r|l}
b & u_x = - u_{\xi} + u_{\eta} \\
c & u_y = u_{\xi} \\
a & u_{xx} = u_{\xi\xi} - 2 u_{\xi\eta} + u_{\eta\eta} \\
2a & u_{xy} = - u_{\xi\xi} + u_{\xi\eta} \\
a & u_{yy} = u_{\xi\xi}
\end{array}
\)
\end{flushleft}

Пересчёт коэффициентов:
\begin{flushleft}
\(
\begin{array}{rcl}
    u_{\xi} &:& -b + c \\
    u_{\eta} &:& b \\
    u_{\xi\xi} &:& a - 2a + a = 0 \\
    u_{\eta\eta} &:& -2a + 2a = 0 \\
    u_{\xi\eta} &:& a
\end{array}
\)
\end{flushleft}

$ au_{\eta\eta} + bu_{\eta} + (c-b)u_{\xi}+u=0$

\vspace{2mm}
Подстановка $ u = e^{\alpha_1 \xi + \alpha_2 \eta} \cdot v(\xi, \eta) $:
\begin{flushleft}
\(
\begin{array}{r@{}l}
    u_{\xi} &{}= (\alpha_1 v + v_{\xi}) e^{\alpha_1 \xi + \alpha_2 \eta}, \\
    u_{\eta} &{}= (\alpha_2 v + v_{\eta}) e^{\alpha_1 \xi + \alpha_2 \eta}, \\
    u_{\eta\eta} &{}= (\alpha_2^2 v + 2\alpha_2 v_{\xi} + v_{\eta\eta}) e^{\alpha_1 \xi + \alpha_2 \eta}.
\end{array}
\)
\end{flushleft}

Подстановка в уравнение:
\begin{equation*}
    (a(\alpha_2^2 v + 2\alpha_2 v_{\xi} + v_{\eta\eta}) + b (\alpha_2 v + v_{\eta}) + (c - b)(\alpha_1 v + v_{\xi}) + v) e^{\alpha_1 \xi + \alpha_2 \eta}= 0.
\end{equation*}

Группировка по $ v, v_{\xi}, v_{\eta}, v_{\eta\eta} $:
\begin{equation*}
    v\left( a \alpha_2^2 + b \alpha_2 + (c - b) \alpha_1 + 1 \right) + 
    v_{\eta} \left( 2\alpha_2 + b \right) + 
    v_{\xi} (c - b) + 
    v_{\eta\eta} a = 0.
\end{equation*}

\text{Зануляем коэффицинты:}
\begin{equation*}
 \begin{cases}
a\, \alpha_2^2 + b\alpha_2+(c - b) \alpha_1 + 1 = 0 \quad\Longrightarrow\quad \alpha_1 = -\frac{2b^2-ab^2-4}{4(c-b)} \\
2\alpha_2 + b = 0 \quad \quad \quad \quad \quad \quad \quad \quad \quad\Longrightarrow\quad \alpha_2 = -\frac{b}{2},
\end{cases}
\end{equation*}

$av_{\eta\eta} + (c - b)\,v_{\xi} = 0.$

\vspace{2mm}
\textbf{Ответ:} $v_{\eta\eta} + \frac{(c - b)}{a}\,v_{\xi} = 0.$

\vspace{100mm}
\vspace{79mm}
\begin{center}    
    \textbf{Задание 5. №3.5}
\end{center}

Найти общее решение уравнения с постоянными коэффициентами

\begin{equation*}
    u_{yy} - 2u_{xy} + 2u_x - u_y = 4e^x
\end{equation*}

\textbf{Решение:}

Характеристическое уравнение:
\begin{equation*}
    (dx)^2 + 2d(dx dy) = 0
\end{equation*}
\begin{equation*}
    dx (dx + 2 dy) = 0
\end{equation*}

Рассмотрим случаи:
\[
\begin{cases}
    dx = 0 \hspace{11mm} \Rightarrow \quad   x = C \\
    dx = -2dy \quad \Rightarrow \quad x = -2y + C
\end{cases}
\]

Выбираем новые переменные:
\begin{align*}
    \xi &= x,   \quad \quad \quad  \xi_x = 1,  \quad \quad \xi_y = 0 \\
    \eta &= x + 2y, \quad  \eta_x = 1, \quad \quad \eta_y = 2
\end{align*}

Преобразование производных:
\begin{flushleft}
\(
\begin{array}{r|l}
2 & u_x = u_{\xi} + u_{\eta} \\
-1 & u_y = 2u_{\eta} \\
0 & u_{xx} = u_{\xi\xi} + 2 u_{\xi\eta} + u_{\eta\eta} \\
-2 & u_{xy} = 2 u_{\xi\eta} + 2 u_{\eta\eta} \\
1 & u_{yy} = 4 u_{\eta\eta}
\end{array}
\)
\end{flushleft}

Пересчёт коэффициентов:
\begin{flushleft}
\(
\begin{array}{rcl}
    u_{\xi} &:& 2 \\
    u_{\eta} &:& 2-2=0 \\
    u_{\xi\xi} &:& 0 \\
    u_{\eta\eta} &:& -4 \\
    u_{\xi\eta} &:& -4 + 4 = 0
\end{array}
\)
\end{flushleft}
\begin{equation*}
    \begin{aligned}
        -4u_{\xi\eta} + 2u_{\xi} &= 4e^x \\
        u_{\xi\eta} - \frac{1}{2} u_{\xi} &= -e^x \\
        \left( u_{\eta} - \frac{1}{2} u \right)_{\xi} &= -e^x
    \end{aligned}
\end{equation*}

Сделаем замену: $ u_{\eta} - \frac{1}{2} u = v$
\begin{equation*}
v_{\xi}=-e^x
\end{equation*}

Интегрируем:
\begin{equation*}
    v = \int  - e^{\xi} d\xi =  - e^{\xi} + C(\eta)
\end{equation*}

Обратная подставновка:
\begin{equation*}
    u_{\eta} - \frac{1}{2} u= - e^{\xi} + C(\eta)
\end{equation*}

Общее решение неоднородного уравнения представим в виде суммы:
\begin{equation*}
    u = u_{\text{общ}} + u_{\text{частн}}
\end{equation*}

где \( u_{\text{общ}} \) — общее решение однородного уравнения, а \( u_{\text{частн}} \) — частное решение.

Рассмотрим однородное уравнение:
\begin{align*}
     u_{\eta} - \frac{1}{2} u=0 \\
     u_{\eta} =\frac{1}{2} u\\
     \frac{du}{d\eta}=\frac{1}{2}u\\
     \frac{du}{\eta}=\frac{d\eta}{2}
\end{align*}

Решение однородного уравнения:
\begin{equation*}
    \begin{aligned}
     ln(u) &=\frac{1}{2}\eta +C_1(\xi) \\
     u&=e^{\frac{1}{2}\eta +C_1(\xi)} \\
     u&= e^{\frac{1}{2}\eta}C_2(\xi)
    \end{aligned}
\end{equation*}

Найдём частное решение:
\begin{equation*}
    \begin{aligned}
        u_{\eta} &= \frac{1}{2} e^{\frac{1}{2}\eta} C_2 + e^{\frac{1}{2}\eta} C_{2\eta}, \\
        \frac{1}{2} e^{\frac{1}{2}\eta} C_2 + e^{\frac{1}{2}\eta} C_{2\eta} 
        &- \frac{1}{2} e^{\frac{1}{2}\eta} C_2 = - e^{\xi} + C(\eta), \\
        e^{\frac{1}{2}\eta} C_{2\eta} &= - e^{\xi} + C(\eta), \\
        C_{2\eta} &= (C(\eta) - e^{\xi}) e^{-\frac{1}{2}\eta}, \\
        C_2(\xi, \eta) &= \int (C(\eta) - e^{\xi}) e^{-\frac{1}{2}\eta} d\eta.
    \end{aligned}
\end{equation*}

Таким образом, общее решение уравнения:
\begin{equation*}
    u = e^{\frac{1}{2}\eta} ( C_1(\xi) + C_2(\xi, \eta) )
\end{equation*}
\begin{equation*}
     u = e^{\frac{1}{2}(x+2y)} ( C_1(x) + C_2(x, x+2y) )
\end{equation*}

\textbf{Ответ:} $u = e^{\frac{1}{2}(x+2y)} ( C_1(x) + C_2(x, x+2y) ).$

\vspace{51mm}
\begin{center}    
    \textbf{Задание 6. №3.30}
\end{center}

В каждой из областей, где сохраняется тип уравнения, найти общее решение уравнения:

\begin{equation*}
    u_{xx} - 2\sin{x} u_{xy} - (3 + \cos^2{x})u_{yy} - \cos{x} u_{y} = 0
\end{equation*}

\textbf{Решение:}

Характеристическое уравнение:
\begin{equation*}
    (dy)^2 + 2\sin{x} (dx \cdot dy) - (3 + \cos^2{x}) (dx)^2 = 0
\end{equation*}
\begin{equation*}
   \left( \frac{dy}{dx} \right)^2 + 2\sin{x} \left( \frac{dy}{dx} \right)- (3 + \cos^2{x}) = 0
\end{equation*}

Подстановка $t = \frac{dy}{dx}$:
\begin{equation*}
    t^2 + 2\sin{x} t - (3 + \cos^2{x}) = 0
\end{equation*}

\begin{equation*}
    D = 4\sin^2{x} + 4\cdot(3 + \cos^2{x}) = 16 \Rightarrow \text{уравнение гиперболическое}
\end{equation*}

Корни уравнения:
\begin{equation*}
    t_{1,2} = -\sin{x} \pm 2
\end{equation*}

Рассмотрим уравнения характеристик:
\[
\begin{cases}
    dy = (-\sin{x} + 2)dx, \\
    dy = (-\sin{x} - 2)dx.
\end{cases}
\]

Интегрируя, получаем:
\[
\begin{cases}
    y = -\cos{x} + 2x + C_1, \\
    y = -\cos{x} - 2x + C_2.
\end{cases}
\]

Выбираем новые переменные:
\begin{align*}
   \xi &= y - \cos{x} + 2x, \quad \xi_x = \sin{x} + 2, \quad \xi_y = 1, \quad \xi_{xx} = \cos{x}, \\
   \eta &= y - \cos{x} - 2x, \quad \eta_x = \sin{x} - 2, \quad \eta_y = 1,  \quad \eta_{xx} = \cos{x}.
\end{align*}

Преобразование производных:
\begin{flushleft}
\(
\begin{array}{r|l}
0 & u_x = (\sin{x} + 2) u_{\xi} + (\sin{x} - 2) u_{\eta} \\
-\cos x & u_y = u_{\xi} + u_{\eta} \\
1 & u_{xx} = (\sin{x} + 2)^2 u_{\xi\xi} + 2(\sin{x} + 2)(\sin{x} - 2) u_{\xi\eta} + (\sin{x} - 2)^2u_{\eta\eta} + \\ & + \cos x u_{\xi} + \cos x u_{\eta}\\
-2 \sin x & u_{xy} = (\sin{x} + 2) u_{\xi\xi} + 2\sin{x} u_{\xi\eta} + (\sin{x} - 2) u_{\eta\eta}  \\
-(3+\cos^2 x) & u_{yy} = u_{\xi\xi} + 2 u_{\xi\eta} + u_{\eta\eta}
\end{array}
\)
\end{flushleft}

Пересчёт коэффициентов:
\begin{flushleft}
\(
\begin{array}{rcl}
    u_{\xi} &:& -\cos x + \cos x = 0 \\
    u_{\eta} &:& -\cos x + \cos x = 0  \\
    u_{\xi\xi} &:& (\sin{x} + 2)^2 - 2\sin{x} (\sin{x} + 2) - (3 + \cos^2{x})=0 \\
    u_{\xi\eta} &:& 2(\sin{x} + 2)(\sin{x} - 2) - 4\sin^2{x} - 2(3 +\cos^2{x})=-16\\
    u_{\eta\eta} &:& (\sin{x} - 2)^2 - 2\sin{x} (\sin{x} - 2) - (3 + \cos^2{x})=0
\end{array}
\)
\end{flushleft}

Подставляя, получаем:
\begin{equation*}
    u_{\xi\eta} = 0
\end{equation*}
\begin{equation*}
    u_{\xi} = C(\xi)
\end{equation*}
\begin{equation*}
    u = \int C(\xi) \, d\xi
\end{equation*}
\begin{equation*}
    u = C_1(\xi) \, + C_2(\eta)
\end{equation*}
\textbf{Ответ:} $u = C_1(y - \cos{x} + 2x) + C_2(y - \cos{x} - 2x)$.

\vspace{100mm}
\vspace{100mm}
\vspace{20mm}
\begin{center}    
\noindent \textbf{Задание 7}
\end{center}

Решить задачу Коши:

\begin{equation*}\label{eq:orig}
    x\,u_{xx} + (x+y)\,u_{xy} + y\,u_{yy}=0,\quad x>0,\;y>0,
\end{equation*}
\[
u \Big|_{y=\frac{1}{x}} = x^3, \quad \quad  u_{y} \Big|_{y=\frac{1}{x}} = -x^4
\]

\textbf{Решение:}

Характеристическое уравнение:
\[
x\left({dy}\right)^2 - 2(x+y)\left({dy}\cdot{dx}\right) + y\left({dx}\right)^2=0.
\]
\[
x\left(\frac{dy}{dx}\right)^2 - 2(x+y)\left(\frac{dy}{dx}\right) + y=0.
\]
Подстановка $t = \frac{dy}{dx}$:
\[
xt^2 - 2(x+y)t + y=0.
\]
Дискриминант $D = (x+y)^2 - 4xy = (x-y)^2> 0$, следовательно, уравнение \textbf{гиперболического типа}.

Решение уравнения:
\begin{equation*}
    \begin{cases}
    t_1 = \frac{y}{x} \\
    t_2 = 1
    \end{cases}
\end{equation*}

Соответствующие характеристические переменные:
\begin{align*}
    &t_1 = \frac{y}{x}: \quad y = xC_1 \\
    &t_2 = 1: \quad y = x + C_2
\end{align*}

Выбираем новые переменные:
\begin{align*}
    \xi &= \frac{y}{x} , \quad \xi_x = -\frac{y}{x^2},  \quad \xi_y = \frac{1}{x},   \quad  \xi_{xx} = \frac{2y}{x^3},  \quad  \xi_{xy} = -\frac{1}{x^2} \\
    \eta &= y - x, \quad \eta_x = -1,  \quad \eta_y = 1
\end{align*}

Преобразование производных:
\begin{flushleft}
\(
\begin{array}{r|l}
0 & u_x = -\frac{y}{x^2} u_{\xi} - u_{\eta} \\
0 & u_y =\frac{1}{x} u_{\xi} + u_{\eta} \\
x & u_{xx} = \frac{y^2}{x^4} u_{\xi\xi} + \frac{2y}{x^2} u_{\xi\eta}  + u_{\eta\eta} + \frac{2y}{x^3} u_{\xi} \\
(x+y) & u_{xy} = -\frac{y}{x^3} u_{\xi\xi} + u_{\xi\eta} (-\frac{y}{x^2} - \frac{1}{x}) - u_{\eta\eta} - \frac{1}{x^2} u_{\xi} \\
y & u_{yy} = \frac{1}{x^2} u_{\xi\xi} + \frac{2}{x} u_{\xi\eta}  + u_{\eta\eta} 
\end{array}
\)
\end{flushleft}

Пересчёт коэффициентов:
\begin{flushleft}
\(
\begin{array}{rcl}
u_{\xi} & : & \frac{2y}{x^3}x-\frac{1}{x^2}(x+y)= \frac{(y-x)}{x^2}\\
u_{\eta} & : & 0 \\
u_{\xi\xi} & : & \frac{y^2}{x^3}-\frac{y(x+y)}{x^3}+\frac{y}{x^2} = 0\\
u_{\xi\eta} & : &  \frac{2y}{x}+(x+y)(-\frac{y}{x^2} - \frac{1}{x}) + \frac{2y}{x} = -\frac{(y-x)^2}{x^2}\\
u_{\eta\eta} & : & x - (x+y) + y = 0
\end{array}
\)
\end{flushleft}
Обратная замена переменных:
\begin{align*}
    x &= \frac{\eta}{\xi - 1}, \quad \quad y = \frac{\xi\eta}{\xi - 1}
\end{align*}
Имеем:
\begin{align*}
    -\frac{(y-x)^2}{x^2}u_{\xi\eta} + \frac{(y-x)}{x^2}u_{\xi} &= 0 \\
    u_{\xi\eta} - \frac{1}{y - x}u_{\xi} &= 0 \\
    u_{\xi\eta} - \frac{1}{\eta}u_{\xi} &= 0
\end{align*}
Замена $u_{\xi}=v$:
\begin{equation*}
    \frac{dv}{d\eta} =\frac{v}{\eta}
\end{equation*}
\begin{equation*}
    v_{\eta} - \frac{1}{\eta}v = 0
\end{equation*}
\begin{equation*}
\ln v =\ln \eta + C
\end{equation*}
\begin{equation*}
v =\eta \cdot C(\xi)
\end{equation*}
Тогда:
\begin{equation*}
    u_{\xi} = \eta \cdot C(\xi)
\end{equation*}
\begin{equation*}
    u = \int \eta \cdot C(\xi) d\xi + C(\eta)
\end{equation*}
\begin{equation*}
    u = \eta \cdot C_1(\xi)+ C_2(\eta)
\end{equation*}
Обратная замена:
\begin{equation*}
    u = (y-x) \cdot C_1(\frac{y}{x})+ C_2(y-x)
\end{equation*}
Используя начальные условия, подставим $ y = \frac{1}{x} $ в выражение для $ u(x,y) $:
\begin{equation*}
    \left( \frac{1}{x} - x \right) C_1 \left( \frac{1}{x^2} \right) + C_2 \left( \frac{1}{x} - x \right) =x^3
\end{equation*}
Обозначим
\begin{equation*}
    \xi = y - x = \frac{1 - x^2}{x}, \quad \eta = \frac{y}{x} = \frac{1}{x^2}.
\end{equation*}
Тогда уравнение принимает вид:
\begin{equation*}
    \xi C_1(\eta) + C_2(\xi) = x^3
\end{equation*}
Второе условие даёт производную:
\begin{equation*}
     C_1(\eta) + \frac{1 - x^2}{x^2} C_1'(\eta) + C_2'(\xi) = -x^4
\end{equation*}
Используем замену переменных $ s = \xi $, $ t = \eta $ и выразим $ x $ через $ t $:
\begin{equation*}
    x = \frac{1}{\sqrt{t}}, \quad s = \frac{t - 1}{\sqrt{t}}
\end{equation*}
Запишем уравнения в терминах $ t $:
\begin{align*}
    C_2(s) + s C_1(t) &= \frac{1}{t^{3/2}},\\
    C_2'(s) + C_1(t) + (t - 1) C_1'(t) &= -\frac{1}{t^2}
\end{align*}
Решим систему уравнений.

1. Выразим $ C_2(s) $:
   \begin{equation*}
       C_2(s) = \frac{1}{t^{3/2}} - s C_1(t).
   \end{equation*}
   
2. Подставим это выражение:
   \begin{equation*}
       \left( \frac{1}{t^{3/2}} - s C_1(t) \right)' + C_1(t) + (t - 1) C_1'(t) = -\frac{1}{t^2}.
   \end{equation*}
   
   Так как $ s = \frac{t - 1}{\sqrt{t}} $, имеем $ ds/dt = \frac{(t - 1)' \sqrt{t} - (t - 1) (\sqrt{t})'}{t} = \frac{\sqrt{t} - \frac{t - 1}{2 \sqrt{t}}}{t} $, что упрощается до $ ds/dt = \frac{t + 1}{2t \sqrt{t}} $.

3. Производная $ C_2(s) $ будет:
   \begin{equation*}
       C_2'(s) = \left( \frac{1}{t^{3/2}} - s C_1(t) \right)' = -\frac{3}{2} t^{-5/2} - \frac{t + 1}{2t \sqrt{t}} C_1(t) - s C_1'(t).
   \end{equation*}
   
   Подставляя это, получаем
   \begin{equation*}
       -\frac{3}{2} t^{-5/2} - \frac{t + 1}{2t \sqrt{t}} C_1(t) - s C_1'(t) + C_1(t) + (t - 1) C_1'(t) = -\frac{1}{t^2}.
   \end{equation*}

4. Переписываем уравнение:
   \begin{equation*}
       (t - 1 - s) C_1'(t) + \left(1 - \frac{t + 1}{2t \sqrt{t}} \right) C_1(t) = -\frac{1}{t^2} + \frac{3}{2} t^{-5/2}.
   \end{equation*}
   
   Так как $ s = \frac{t - 1}{\sqrt{t}} $, имеем $ t - 1 - s = \frac{(t - 1)(\sqrt{t} - 1)}{\sqrt{t}} $.
   
5. Разделяем переменные и интегрируем уравнение относительно $ C_1(t) $:
   \begin{equation*}
       C_1'(t) = \frac{1}{t^2} - \frac{1}{(t - 1)^2}.
   \end{equation*}
   
   Интегрируя, находим
   \begin{equation*}
       C_1(t) = \frac{1}{t - 1} - \frac{1}{t}.
   \end{equation*}
   
6. Подставляем $ C_1(t) $ в уравнение для $ C_2(s) $ и находим $ C_2(s) = 0 $.

Подставляя найденные функции, получаем решение:
\begin{equation*}
    u(x,y) = \frac{x^2}{y}.
\end{equation*}
\textbf{Ответ:} $u =\frac{x^2}{y}$.


\vspace{14mm}
\begin{center}    
    \textbf{Задание 8. №5}
\end{center}

Решить задачу Гурса:

\begin{equation*}
    u_{xx} + 6 u_{xy} + 5u_{yy} = 0, \quad y<5x, \quad  x>0
\end{equation*}
\[
u \Big|_{y=0} = \cos x, \quad \quad  u \Big|_{y=5x} = \cos 2x
\]
\textbf{Решение:}

Характеристическое уравнение:
\begin{equation*}
     (dy)^2 + 6 (dx \cdot dy) + 5(dx)^2=0
\end{equation*}
\begin{equation*}
\left( \frac{dy}{dx} \right)^2 + 6 \left( \frac{dy}{dx} \right) + 5=0
\end{equation*}

Подстановка $t = \frac{dy}{dx}$:
\begin{equation*}
    t^2 + 6 t + 5=0
\end{equation*}

\begin{equation*}
    D = 36 - 20 = 16 \Rightarrow \text{уравнение гиперболическое}
\end{equation*}

Корни уравнения:
\begin{equation*}
    \begin{cases}
    t_1 = 1 \\
    t_2 = \frac{1}{3}
    \end{cases}
\end{equation*}

Рассмотрим уравнения характеристик:
\[
\begin{cases}
    dy = dx, \\
    dy = 5dx.
\end{cases}
\]

Интегрируя, получаем:
\[
\begin{cases}
    y = x + C_1, \\
    y = 2x + C_2.
\end{cases}
\]

Выбираем новые переменные:
\begin{align*}
    \xi &= y - 5x, \quad \xi_x = -5,  \quad \quad \xi_y = 1 \\
    \eta &= y - x, \quad \eta_x =- 1,  \quad \quad \eta_y = 1
\end{align*}

Преобразование производных:
\begin{flushleft}
\(
\begin{array}{r|l}
0 & u_x = -5u_{\eta} - u_{\xi} \\
0 & u_y = u_{\eta} + u_{\xi} \\
1 & u_{xx} = 25 u_{\xi\xi} + 10u_{\xi\eta} + u_{\eta\eta}\\
6 & u_{xy} = -5 u_{\xi\xi} - 4u_{\xi\eta} - u_{\eta\eta}\\
5 & u_{xy} = u_{\xi\xi} + 2 u_{\xi\eta}+u_{\eta\eta}
\end{array}
\)
\end{flushleft}

Пересчёт коэффициентов:
\begin{flushleft}
\(
\begin{array}{rcl}
u_{\xi} & : & 0 \\
u_{\eta} & : & 0 \\
u_{\xi\xi} & : & 25 - 30 + 5 = 0\\
u_{\xi\eta} & : & 10 -24 + 10 = -4\\
u_{\eta\eta} & : &  1 - 6 + 5 = 0
\end{array}
\)
\end{flushleft}

Подставляя, получаем:
\begin{equation*}
    -4u_{\xi\eta} = 0
\end{equation*}
\begin{equation*}
    u_{\xi\eta} = 0
\end{equation*}
\begin{equation*}
    u_{\xi} = C(\xi)
\end{equation*}
\begin{equation*}
    u = \int C(\xi) \, d\xi
\end{equation*}
\begin{equation*}
    u = C_1(\xi) \, + C_2(\eta)
\end{equation*}
\begin{equation*}
    u = C_1(y-5x) \, + C_2(y-x)
\end{equation*}

Используя начальные условия:
\begin{align*}
    &C_1(-5x) \, + C_2(-x) = \cos x \\
    &C_1(0) \, + C_2(4x) = \cos 2x
\end{align*}

Выразим $C_1$ и $C_2$:
\begin{align*}
    &C_1(z)=\cos (-\frac{z}{5}) - C_2(\frac{z}{5})\\
    &C_2(z) = \cos (\frac{z}{2}) + C_1(0)
\end{align*}

Тогда имеем:
\begin{align*}
    u &=\cos (\frac{5x-y}{5}) - C_2(\frac{y-5x}{5}) + \cos (\frac{y-x}{2}) - C_1(0)\\
    u &=\cos (\frac{5x-y}{5}) - \cos (\frac{y-5x}{10}) +C_1(0) + \cos (\frac{y-x}{2}) - C_1(0)\\
    u &=\cos (\frac{5x-y}{5}) - \cos (\frac{y-5x}{10}) + \cos (\frac{y-x}{2}) \\
\end{align*}

\textbf{Ответ:} $u =\cos (\frac{5x-y}{5}) - \cos (\frac{y-5x}{10}) + \cos (\frac{y-x}{2})$.

\vspace{90mm}
\begin{center}    
    \textbf{Задание 9. №2.5}
\end{center}

Привести к каноническому виду и упростить уравнение:

\begin{equation*}
    u_{xx} + u_{yy} + u_{zz} + 6u_{xy} + 2u_{xz} + 2u_{yz} + 2u_{x} + 2u_{y} + 2u_{z} + 4u = 0
\end{equation*}
\textbf{Решение:}

\[
\Phi (t) = t_1^2 + t_2^2 + t_3^2 + 6t_1t_2 + 2t_1t_3 + 2t_2t_3 =
\]
\[
= (t_1^2 + 2t_1(3t_2 + t_3) + (3t_2 + t_3)^2) - (3t_2 + t_3)^2 + t_2^2+ t_3^2+ 6t_1t_2 + 2t_2t_3=
\]
\[
= (t_1 + 3t_2 + t_3)^2 - 9t_2^2 - 6t_2t_3 - t_3^2 + t_2^2 +t_3^2+ 2t_2t_3=
\]
\[
= (t_1 + 3t_2 + t_3)^2 - 8t_2^2 -4t_2t_3=
\]
\[
= (t_1 + 3t_2 + t_3)^2 - (8t_2^2 +2(2\sqrt{2}t_2)(\frac{t_3}{\sqrt{2}})+ \frac{t_3^2}{2})+\frac{t_3^2}{2}=
\]
\[
= (t_1 + 3t_2 + t_3)^2 - (\sqrt{8}t_2 + \frac{t_3}{\sqrt{2}})^2+(\frac{t_3}{\sqrt{2}})^2=\tau_1^2 - \tau_2^2 + \tau_3^2
\]
Замена $\tau_1, \,\tau_2, \,\tau_3$:
\begin{align*}
   &\tau_1 = t_1 + 3t_2 + t_3\\
   &\tau_2 = \sqrt{8}t_2 + \frac{t_3}{2} \\
   &\tau_3 =\frac{t_3}{2}
\end{align*}
Обратная замена $t_1, \,t_2, \,t_3$:
\begin{align*}
   &t_1 = \tau_1 - \frac{3}{\sqrt{8}}\tau_2 -\frac{1}{\sqrt{8}} \tau_3\\
   &t_2 = \frac{\tau_2-\tau_3}{2\sqrt{2}} \\
   &t_3 =\sqrt{2}\tau_3
\end{align*}
Матрица преобразования:
\[
B = 
\begin{pmatrix}
1 & -\frac{3}{\sqrt{8}} & -\frac{1}{\sqrt{8}} \\
0 & \frac{1}{2\sqrt{2}} & -\frac{1}{2\sqrt{2}} \\
0 & 0 & \sqrt{2}
\end{pmatrix}
\]

\[
B^{T} =
\begin{pmatrix}
1 & 0 & 0 \\
-\frac{3}{\sqrt{8}} & \frac{1}{2\sqrt{2}} & 0 \\
-\frac{1}{\sqrt{8}} & -\frac{1}{2\sqrt{2}} & \sqrt{2}
\end{pmatrix}
\]
Замена переменных:
\[
\begin{pmatrix}
y_1 \\
y_2 \\
y_3
\end{pmatrix}
=
B^{T}
\begin{pmatrix}
x \\
y \\
z
\end{pmatrix}
\Rightarrow
\begin{pmatrix}
y_1 \\
y_2 \\
y_3
\end{pmatrix}
=
\begin{pmatrix}
1 & 0 & 0 \\
-\frac{3}{\sqrt{8}} & \frac{1}{2\sqrt{2}} & 0 \\
-\frac{1}{\sqrt{8}} & -\frac{1}{2\sqrt{2}} & \sqrt{2}
\end{pmatrix}
\begin{pmatrix}
x \\
y \\
z
\end{pmatrix}
\]
Тогда $y_1, y_2, y_3$:
\begin{align*}
   &y_1 = x\\
   &y_2 = -\frac{3}{\sqrt{8}}x + \frac{1}{2\sqrt{2}}y\\
   &y_3 =-\frac{1}{\sqrt{8}}x-\frac{1}{2\sqrt{2}}y+ \sqrt{2}z
\end{align*}
Выразим $x, y, z$:
\begin{align*}
   &x = y_1\\
   &y = 2\sqrt{2}(y_2 +\frac{3}{\sqrt{8}}y_1)\\
   &z =\frac{1}{\sqrt{2}}(y_3+y_2+\sqrt{2}y_1)
\end{align*}
Преобразование производных:
\begin{flushleft}
\(
\begin{array}{r|l}
2 & u_x = u_{y_1} - \frac{3}{\sqrt{8}}u_{y_2} - \frac{1}{\sqrt{8}}u_{y_3} \\
2 & u_y = \frac{1}{2\sqrt{2}}u_{y_2} - \frac{1}{2\sqrt{2}} u_{y_3}  \\
2 & u_z = \sqrt{2}u_{y_3}  \\
1 & u_{xx} = u_{y_1y_1} - \frac{3}{\sqrt{8}}u_{y_1y_2} - \frac{1}{\sqrt{8}}u_{y_1y_3} - \frac{3}{\sqrt{8}}u_{y_2y_1} +\frac{9}{8} u_{y_2y_2} + \frac{3}{8}u_{y_2y_3} -\frac{1}{\sqrt{8}} u_{y_3y_1} + \frac{3}{8}u_{y_3y_2} + \\& + \frac{1}{8} u_{y_3y_3} = u_{y_1y_1} +\frac{9}{8} u_{y_2y_2} +\frac{1}{8} u_{y_3y_3}  - \frac{6}{\sqrt{8}}u_{y_1y_2} - \frac{2}{\sqrt{8}}u_{y_1y_3} + \frac{3}{4}u_{y_2y_3} \\
1 & u_{yy} = \frac{1}{8}u_{y_2y_2} - \frac{1}{8}u_{y_2y_3} - \frac{1}{8}u_{y_2y_3} + \frac{1}{8} u_{y_3y_3} = \frac{1}{8}u_{y_2y_2} + \frac{1}{8} u_{y_3y_3} - \frac{1}{4}u_{y_2y_3}  \\
1 & u_{zz} = 2u_{y_3y_3} \\
6 & u_{xy} = \frac{1}{2\sqrt{2}}u_{y_2y_1} - \frac{3}{8} u_{y_2y_2} - \frac{1}{8}u_{y_2y_3} - \frac{1}{2\sqrt{2}}u_{y_3y_1} + \frac{3}{8}u_{y_2y_3} + \frac{1}{8} u_{y_3y_3} =\\&= - \frac{3}{8} u_{y_2y_2} + \frac{1}{8} u_{y_3y_3} + \frac{1}{2\sqrt{2}}u_{y_1y_2}  - \frac{1}{2\sqrt{2}}u_{y_1y_3} + \frac{1}{4}u_{y_2y_3} \\
2 & u_{yz} = \sqrt{2} \cdot \frac{1}{2\sqrt{2}} u_{y_3y_2} - \sqrt{2} \cdot \frac{1}{2\sqrt{2}} u_{y_3y_3} = \frac{1}{2} u_{y_3y_2} - \frac{1}{2} u_{y_3y_3}  \\
2 & u_{xz} = \sqrt{2} u_{y_3y_1} - \frac{3}{2}u_{y_3y_2} - \frac{1}{2}u_{y_3y_3} =  - \frac{1}{2}u_{y_3y_3} + \sqrt{2} u_{y_1y_3} - \frac{3}{2}u_{y_2y_3} 
\end{array}
\)
\end{flushleft}

Пересчет коэффициентов:
\begin{flushleft}
\(
\begin{array}{rcl}
u_{y_1} & : & 2 \\
u_{y_2} & : & -\frac{3}{\sqrt{2}} + \frac{1}{\sqrt{2}} = -\sqrt{2} \\
u_{y_3} & : & -\frac{1}{\sqrt{2}} - \frac{1}{\sqrt{2}} + 2\sqrt{2}= \sqrt{2}\\
u_{y_1y_1} & : & 1 \\
u_{y_2y_2} & : & \frac{9}{8} + \frac{1}{8} - \frac{9}{4}= -1 \\
u_{y_3y_3} & : & \frac{1}{8} + \frac{1}{8} + 2 + \frac{3}{4} - 1 -1 = 1\\
u_{y_1y_2} & : & -\frac{6}{\sqrt{8}} +\frac{6}{\sqrt{8}} = 0\\
u_{y_1y_3} & : & -\frac{2}{\sqrt{8}} -\frac{6}{\sqrt{8}} + 2\sqrt{2} = 0 \\
u_{y_2y_3} & : & \frac{3}{4} - \frac{1}{4} + \frac{6}{4} + 1 - 3 = 0 
\end{array}
\)
\end{flushleft}

Подставляя, получаем:
\begin{equation*}
    u_{y_1y_1} - u_{y_2y_2} + u_{y_3y_3} + 2u_{y_1} - \sqrt{2} u_{y_2} + \sqrt{2}u_{y_3} + 4u = 0
\end{equation*}

Введем функцию $u =e^{\alpha_1y_1 + \alpha_2y_2 + \alpha_3y_3}v(y_1,y_2,y_3)$ и обозначим $\alpha_1y_1 + \alpha_2y_2 + \alpha_3y_3 = -$, \\ $v(y_1,y_2,y_3) = v$,  тогда $u =e^{-}v$. Посчитаем производные:
\begin{flushleft}
\(
\begin{array}{r@{}l}
    u_{y_1} &{}= (\alpha_1v + v_{}y_1) e^{-} \\
    u_{y_2} &{}= (\alpha_2v + v_{}y_2) e^{-} \\
    u_{y_3} &{}= (\alpha_2v + v_{}y_2) e^{-} \\
    u_{y_1y_1} &{}= (\alpha_1^2v + 2\alpha_1v_{y_1} + v_{y_1y_1}) e^{-} \\
    u_{y_2y_2} &{}= (\alpha_2^2v + 2\alpha_2v_{y_2} + v_{y_2y_2}) e^{-}  \\
    u_{y_3y_3} &{}= (\alpha_3^2v + 2\alpha_3v_{y_3} + v_{y_3y_3}) e^{-} 
\end{array}
\)
\end{flushleft}
Подставим в уравнение:
\begin{align*}
    &\left( \alpha_1^2 v + 2\alpha_1 v_{y_1} + v_{y1y1}  
    - \alpha_2^2 v - 2\alpha_2 v_{y_2} - v_{y_2 y_2} \right. \\ 
    &\quad + \alpha_3^2 v + 2 \alpha_3 v_{y_3} + v_{y_3 y_3} 
    + 2 \alpha_1 v + 2 v_{y_1} \\ 
    &\quad \left. - \sqrt{2} \alpha_2 v - \sqrt{2} v_{y_2} 
    + \sqrt{2} \alpha_3 v + 2 v_{y_3} + 4 v \right) e^{-t} = 0.
\end{align*}
Коэффициенты при $v, v_{y_1}, v_{y_2}, v_{y_3}, v_{y_1y_1}, v_{y_2y_2}, v_{y_3y_3},$:
\begin{flushleft}
\(
\begin{array}{rcl}
v & : & \alpha_1^2 - \alpha_2^2 + \alpha_3^2 - 2\alpha_1 - \sqrt{2}\alpha_2 + \sqrt{2} \alpha_3 + 4 \\
v_{y_1} & : & 2\alpha_1 + 2\\
v_{y_2} & : & -2\alpha_2 + \sqrt{2}\\
v_{y_3} & : & 2\alpha_3 + \sqrt{2} \\
v_{y_1y_1} & : & 1 \\
v_{y_2y_2} & : & -1\\
v_{y_3y_3} & : & -1
\end{array}
\)
\end{flushleft}
Возьмём $\alpha_1=-1, \alpha_2=-\frac{1}{\sqrt{2}}, \alpha_3=-\frac{1}{\sqrt{2}}$, тогда уравнение примет вид:
\begin{equation*}
    3v + v_{y_1y_1} - v_{y_2y_2} + v_{y_3y_3}= 0
\end{equation*}

\textbf{Ответ:} $v_{y_1y_1} - v_{y_2y_2} + v_{y_3y_3} + 3v= 0$.

\end{document}
